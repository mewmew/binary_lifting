% === [ Examples of Disassembly Failure Modes ] ================================

% TODO: check when to use disassembly vs. disassembler.

\chapter{Examples of Disassembly Failure Modes}
\label{app:examples_of_disassembly_failure_modes}

% --- [ Linear Sweap Disassembly Failure Modes ] -------------------------------

% Data in Code Segment

\section{Linear Sweap Disassembly Failure Modes}
\label{app:linear_sweap_disassembly_failure_modes}

These examples illustrate the linear sweap disassembly failure mode where data is not properly separated from code.

% ~~~ [ String Literal in Code ] ~~~~~~~~~~~~~~~~~~~~~~~~~~~~~~~~~~~~~~~~~~~~~~~

% TODO: use lstlisting LaTeX magic to highlight lines with incorrect
% disassembly.

\subsection{String Literal in Code}

\lstinputlisting[linebackgroundcolor={\btLstHL{9}}, language=nasm, style=nasm, tabsize=4, linerange={5-20}, caption={Assembly example with data in code.}, label={lst:data_in_code_asm}]{inc/2_literature_review/1_disassembly/data_in_code.asm}

\lstinputlisting[linebackgroundcolor={\btLstHL{5-15}}, language=nasm, style=nasm, tabsize=4, caption={Failure to disassemble listing \ref{lst:data_in_code_asm} using linear sweap disassembler (\texttt{objdump}) due to data in code segment. Incorrect disassembly at lines 5-15.}, label={lst:data_in_code_linear_fail}]{inc/2_literature_review/1_disassembly/data_in_code_linear_fail.objdump.disasm}

% ~~~ [ Jump Table in Code Segment ] ~~~~~~~~~~~~~~~~~~~~~~~~~~~~~~~~~~~~~~~~~~~

\subsection{Jump Table in Code Segment}

\lstinputlisting[linebackgroundcolor={\btLstHL{11-14}}, language=nasm, style=nasm, tabsize=4, linerange={13-43}, caption={Assembly example with jump table in code.}, label={lst:jump_table_in_code_asm}]{inc/2_literature_review/1_disassembly/jump_table_in_code.asm}

\lstinputlisting[linebackgroundcolor={\btLstHL{8-13}}, language=nasm, style=nasm, tabsize=4, caption={Failure to disassemble listing \ref{lst:jump_table_in_code_asm} using linear sweap disassembler (\texttt{objdump}) due to jump table in code segment. Incorrect disassembly at lines 8-13.}, label={lst:jump_table_in_code_linear_fail}]{inc/2_literature_review/1_disassembly/jump_table_in_code_linear_fail.objdump.disasm}

% --- [ Recursive Descent Disassembly Failure Modes ] --------------------------

\section{Recursive Descent Disassembly Failure Modes}
\label{app:recursive_descent_disassembly_failure_modes}

These examples illustrate the recursive descent disassembly failure mode where the instructions of a fake branch target are decoded. As an anti-reverse engineering technique, a fake branch target is used in a conditional branch where the condition is set to always target the non-fake branch.

% ~~~ [ Fake Branch Target ] ~~~~~~~~~~~~~~~~~~~~~~~~~~~~~~~~~~~~~~~~~~~~~~~~~~~

\subsection{Fake Branch Target}

\lstinputlisting[linebackgroundcolor={\btLstHL{5-6}}, language=nasm, style=nasm, tabsize=4, linerange={3-11}, caption={Assembly example with fake branch target.}, label={lst:fake_branch_target_asm}]{inc/2_literature_review/1_disassembly/fake_branch_target.asm}

\lstinputlisting[linebackgroundcolor={\btLstHL{5-7}}, language=nasm, style=nasm, tabsize=4, caption={Failure to disassemble listing \ref{lst:fake_branch_target_asm} using recursive descent disassembler (\texttt{IDA}) due to fake branch target. Incorrect disassembly at lines 5-7.}, label={lst:fake_branch_target_recursive_fail}]{inc/2_literature_review/1_disassembly/fake_branch_target_recursive_fail.ida.disasm}
