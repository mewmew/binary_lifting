\documentclass[12pt, a4paper]{article}

\usepackage{preamble}

\title{Degree Project Proposal \\ \Large Binary Lifting using Type Information Oracles}
\author{Robin Eklind \\ \url{robinek@kth.se}}

\begin{document}

% === [ Front matter ] =========================================================

\pagenumbering{roman}

% --- [ Title page ] -----------------------------------------------------------

\maketitle

\clearpage

% --- [ Table of contents ] ----------------------------------------------------

\tableofcontents

\clearpage

% === [ Main matter ] ==========================================================

\pagenumbering{arabic}

\section{Preliminary Thesis Title}

\begin{quote}
	\textit{What will be the subject of the degree project?}
\end{quote}

The preliminary title of the degree project is ``Binary Lifting using Type Information Oracles''.

\section{Background/Conditions}

\begin{quote}
	\textit{Why and where will the practical part of the degree project be carried out?}
\end{quote}

The degree project will be carried at at KTH with Roberto Guanciale as supervisor, and the research topic is motivated \todo{foo}.

\begin{quote}
	\textit{``It is a problem that has impact. There is a large amount of closed source code and binary blobs that are developed by third parties. Moreover, in some cases we would like to not rely on correctness of the compiler (or assume that the compiler has not been tampered).''} -- Roberto Guanciale
\end{quote}

\section{Research Question}

\begin{quote}
	\textit{A degree project must examine a specific research/technical question. Provisionally state:}
\end{quote}

\begin{quote}
	\textbf{The QUESTION} that will be examined.
\end{quote}

The research question to be examined in this project is \textit{``Can the principle of separation of concern be leveraged to improve upon the state-of-the-art in binary analysis and binary lifting?''}

\begin{quote}
	\textbf{The RESEARCH AREA} for the project.
\end{quote}

\todo{foo}

\begin{quote}
	\textbf{CONNECTION TO RESEARCH/DEVELOPMENT}: Describe how the assignment is connected to current research or development. Describe why the question is of interest and to whom?
\end{quote}

\todo{foo}

\begin{quote}
	\textbf{EXAMINATION METHOD}: How shall the specified question be examined?
\end{quote}

\todo{foo}

\begin{quote}
	\textbf{HYPOTHESIS}: What are the possible/probable outcomes of the examination?
\end{quote}

\todo{foo}

\begin{quote}
	\textbf{EVALUATION}: How can one determine if the objective of the degree project has been fulfilled and if the question has been answered adequately?
\end{quote}

\todo{foo}

\section{Background of the Degree Project Student}

\begin{quote}
	\textit{Describe the knowledge (courses and/or experiences) you have that makes this an appropriate assignment for you.}
\end{quote}

For this section, the author will switch to a first person narrative.

For the last 10 years, I've been working as an IT-security consultant; full-time prior to becoming a University student and part-time since. As

\section{Limits/Resources}

\begin{quote}
	\textit{What is already available at the company (or other host institution) in the form of previous projects, software, expertise, etc.?}
\end{quote}

% The description must ensure that there is adequate preparation so that the degree project student does not need to do all their practical work without an established foundation, thus assuring time for conducting the scientific investigation.

\todo{foo}

\section{Eligibility and Study Planning}

\begin{quote}
	\textit{\textbf{Eligibility}: It is your responsibility, as a student, to verify that you are eligible to start your degree project. You must in your assignment description assure that you have completed all courses for the bachelors' degree and at least 60hp completed second cycle courses and that the 60hp include a course in scientific theory and method (e.g. DA2205, DA2210, DH2610, DM2572) and all courses relevant for the thesis.}

	\textit{\textbf{Study planning}: Also, you must list all courses remaining before you may apply for the master's degree and briefly describe how and when you plan to complete those courses.}
\end{quote}

% This is aimed to ensure that the thesis really is one of the last elements of your education.

\todo{foo}

foo \cite{evaluation_of_irs}

% === [ Back matter ] ==========================================================

% --- [ References ] -----------------------------------------------------------

\bibliography{references}

% --- [ Appendices ] -----------------------------------------------------------

%% === [ Appendices ] ===========================================================

\appendix
\setcounter{secnumdepth}{0}
\section{Appendices}
\setcounter{secnumdepth}{3}
\renewcommand{\thesubsection}{\Alph{subsection}}

% TODO: add appendices.

%% === [ Foo ] ==================================================================

\chapter{Foo}

\todo{foo}

%\clearpage
%\input{sections/appendices/b_foo}
%\clearpage


\end{document}
