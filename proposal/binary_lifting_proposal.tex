\documentclass[12pt, a4paper]{article}

\usepackage{preamble}

\title{Degree Project Proposal \\ \Large Lifting Binaries to LLVM IR using Type Information Oracles}
\author{Robin Eklind \\ \input{my_email}}

% ref: https://www.kth.se/social/group/examensarbete-vid-cs/page/exjobb-task-description/

\begin{document}

% === [ Front matter ] =========================================================

\pagenumbering{roman}

% --- [ Title page ] -----------------------------------------------------------

\maketitle

\clearpage

% --- [ Table of contents ] ----------------------------------------------------

\tableofcontents

\clearpage

% === [ Main matter ] ==========================================================

\pagenumbering{arabic}

\section{Preliminary Thesis Title}

\begin{quote}
	\textit{What will be the subject of the degree project?}
\end{quote}

%\todo{update title: using a diverse set of oracles.}

The preliminary title of the degree project is ``Lifting Binaries to LLVM IR using Type Information Oracles''.

\section{Background/Conditions}

\begin{quote}
	\textit{Why and where will the practical part of the degree project be carried out?}
\end{quote}

The degree project will be carried out at KTH with Roberto Guanciale as supervisor, and the research topic is motivated by several use cases, both in academia and industry. In particular, the aim of this research project is to improve upon the state-of-the-art in binary lifting to facilitate binary analysis; a topic that is both under active research in a multitude of academic institutions and that is of great interest to and has real-world impact in the IT-security industry.

Below follows an extract by Roberto Guanciale to motivate working on this research problem:

\begin{quote}
	\textit{``It is a problem that has impact. There is a large amount of closed source code and binary blobs that are developed by third parties. Moreover, in some cases we would like to not rely on correctness of the compiler (or assume that the compiler has not been tampered).''} -- Roberto Guanciale
\end{quote}

%\todo{add ref: G. Balakrishnan and T. Reps, ``WYSINWYX: What you see is not what you execute,'' ACM Transactions on Programming Languages and Systems, vol. 32, no. 6, pp. 23:1–23:84, 2010.}

\section{Research Question}

\begin{quote}
	\textit{A degree project must examine a specific research/technical question. Provisionally state:}
\end{quote}

\begin{quote}
	\textit{\textbf{The QUESTION} that will be examined.}
\end{quote}

The research question(s) to be examined are:

\begin{enumerate}
	\item ``What factors limit the effectiveness and usability of current binary lifters (e.g. MC-Semantics\footnote{MC-Semantics: \url{https://github.com/trailofbits/mcsema}}, Dagger\footnote{Dagger: \url{https://github.com/repzret/dagger}}, Miasm\footnote{Miasm: \url{https://github.com/cea-sec/miasm}}) that restrict the use of their intermediate representation (IR) output in further binary analysis applications (e.g. decompilation, automated vulnerability discovery, validation of concurrent memory access, etc)?''; and
	\item ``Would using Type Information Oracles improve the effectiveness and usability of binary lifters by improving the quality of their IR output, and thus facilitate further binary analysis use cases?''
\end{enumerate}

% \todo{foo} add note about the output usability of the IR. now using a CPU struct, but rather outputing a higher-level IR with function arguments being identified. facilitates further binary analysis.

\begin{quote}
	\textit{\textbf{The RESEARCH AREA} for the project.}
\end{quote}

The research area(s) of the project are binary analysis, IT-security, type analysis, and formal methods.

\begin{quote}
	\textit{\textbf{CONNECTION TO RESEARCH/DEVELOPMENT}: Describe how the assignment is connected to current research or development. Describe why the question is of interest and to whom?}
\end{quote}

The research interest in binary analysis and in particular decompilers date back to the 1960s, when decompilers were used to facilitate the translation of software written for second generation computers (those using transistors) to third generation computers (those using integrated circuits) \cite{reverse_compilation}. Since then binary analysis has seen more intense research interest and a wide range of applications in industry. Today binary analysis is among others used for validating the output of compilers (a tribute to Ken Thompson's \textit{``Reflections on Trusting Trust''} Turing Award Lecture \cite{trusting_trust}), transpiling locating and mitigating security vulnerabilities \cite{state_of_the_art_of_war}, transpiling software from one source language to another target language or from one source platform to another target platform, instrumenting binaries with profiling capabilities or security enforcements (e.g. control flow integrity enforcement) \cite{superset_disassembly}, and analyzing polyglot code bases (those written in multiple programming languages) where the platform-independent IR functions as a shared representation layer for analysis \cite{revgen}.

However, the way that current binary lifters (including MC-Semantics and Dagger) often model instruction semantics and their effect on CPU state is by passing around a pointer to a CPU state structure -- containing fields for each register in the register bank -- as a parameter to functions being invoked. While semantically correct, this approach limits the effectiveness of the output IR as it requires precise pointer aliasing and sophisticated inter-procedural analysis.

% Output as a single function (rev.ng)

\begin{quote}
	\textit{\textbf{EXAMINATION METHOD}: How shall the specified question be examined?}
\end{quote}

The research question(s) will be examined first by conducting an in-depth research literature review, then once a set of state-of-the-art binary lifters have been identified in recent literature, these different binary lifters will be evaluated to assess how they perform on both quantitative and qualitative measures of binary lifting quality. Preliminary quantifying parameters are defined below. The author expects to expand and refine the quantifying parameters following the literature review, especially based on the methodology of evaluation presented in \cite{evaluation_of_irs}.

Quantifying parameters:
\begin{itemize}
	\item quality of output
	\begin{itemize}
		\item number of standard LLVM IR tools that may be used without modifying the output IR
		\begin{itemize}
			\item KLEE (a Symbolic Execution Engine for LLVM IR)
			\item $\dots$
		\end{itemize}
		\item correctness (e.g. percentage of test cases in Coreutils passing in lifted LLVM IR)
		\item size of output (e.g. line count and byte count)
		\item speed of execution of lifted IR
		%\item histogram of instructions used?
	\end{itemize}
	\item ease of use
	\item ease of installation (measured in wall-time to compile and dependencies and tools used for lifting)
	\item performance of tools
\end{itemize}

\begin{quote}
	\textit{\textbf{HYPOTHESIS}: What are the possible/probable outcomes of the examination?}
\end{quote}

A probable outcome of the degree project is that a set of limitations are identified in the current binary lifters available, as well as strengths and sound approaches for binary lifting. In particular, one limiting factor may be the chosen way to model CPU state changes, as outlined in the \textit{CONNECTION TO RESEARCH/DEVELOPMENT} subsection. Other limiting factors will be identified and outlined in the project.

Another outcome of the degree project is the development of binary lifting techniques that mitigate identified limitations, and in doing so strive to improve upon the state-of-the-art in binary lifting.

\begin{quote}
	\textit{\textbf{EVALUATION}: How can one determine if the objective of the degree project has been fulfilled and if the question has been answered adequately?}
\end{quote}

Evaluating the objectives of the degree project should be quite straight forward as the limiting factors may easily be assessed for different binary lifters (e.g. whether their IR output can be used as input to standard LLVM IR analysis tools). Furthermore, the same evaluation methodologies that would be used for current binary lifters could also be applied to the to-be-developed binary lifter technologies of the degree project, the quantitative performance measurement of which could serve as one evaluation metric. Furthermore, the limitations would be documented as part of the degree project report, which would constitute a major project deliverable.

\section{Background of the Degree Project Student}

\begin{quote}
	\textit{\textit{Describe the knowledge (courses and/or experiences) you have that makes this an appropriate assignment for you.}}
\end{quote}

For this section, the author of the degree project proposal will switch to a first person narrative.

For the last 10 years, I've been working as an IT-security consultant; full-time prior to becoming a University student and part-time since. To assess the security of IT-systems of customers, I have together with my colleges conducted a range of penetration tests, malware forensic investigations, reverse engineering of software components and network protocols. To facilitate this work, we use state-of-the-art tools available for binary analysis and decompilation (including but not limited to the interactive disassembler IDA Pro, the Hex-Rays decompiler, a range of debuggers, execution tracers and binary analysis tools such as MC-Semantics). While these tools have facilitated our work tremendously, there are also been weak points that made themselves apparent over the years. For these reasons, I have conducted several prior research projects trying to improve upon the state-of-the-art in this areas. To date, I've written my Bachelor's degree project on the topic of \textit{``Compositional Decompilation using LLVM IR''} \cite{compositional_decompilation}, an individual research project titled \textit{``Evaluation of Methods for Effective Control Flow Recovery''} supervised by Christian Schulte, an individual research project titled \textit{``Type Analysis of Low-level Code''}\footnote{Type Analysis of Low-level Code: \url{https://github.com/decomp/doc/raw/master/report/type_analysis/type_analysis.pdf}} as part of the course \textit{Principles of Programming Languages} given by Philipp Haller. The Master's degree research project presented in this proposal is the natural continuation of this prior research, and each of these works complement one another.

At KTH, UU and University of Portsmouth I've taken the following courses that are particularly relevant for this Master's degree project on binary analysis:

\begin{itemize}
	\item \textbf{1DL321} Compiler Design I (Uppsala University)
	\item \textbf{1DL420} Compiler Design Project (Uppsala University)
	\item \textbf{1DL441} Combinatorial Optimisation using Constraint Programming (Uppsala University)
	\item \textbf{U23524} Malware Forensics (University of Portsmouth)
	\item \textbf{DD2481} Principles of Programming Languages (KTH)
	\item \textbf{DD2452} Formal Methods (KTH)
\end{itemize}

\section{Limits/Resources}

\begin{quote}
	\textit{\textit{What is already available at the company (or other host institution) in the form of previous projects, software, expertise, etc.?}}
\end{quote}

As part of a preparation for the literature review, at least one research paper \cite{sound_transpilation_from_binary_to_ir} has been identified that directly relates to the research to be conducted within this degree project, the author of which is Roberto Guanciale, who will be supervising this project. As such, the host institution can definitely be considered to have expertise in the research area of inquiry, with research groups conducting active research in the same domain.

For reference, the preparation for the literature review has resulted in the following preliminary list of research papers to assess \cite{graph_based_ir,retargetable_binary_translator,dynamic_binary_translation,uqbt_binary_translator,retargetable_static_binary_translator,ssa_for_decompilation,valgrind,bitblaze,architecture_independent_binary_analysis_and_transformation,arm_analysis_using_llvm,bap,revgen,wartell_rewriting_x86_binaries,barf,singled_graph_disassembly,dbill,instruction_idiom_detection,graph_based_higher_order_ir,reconstruction_of_instruction_idioms,extracting_instruction_semantics_via_symbolic_exectuion_of_code_generators,software_transformation,state_of_the_art_of_war,function_recovery,bin2llvm,sound_transpilation_from_binary_to_ir,cast_study_llvm_suitable_for_binary_analysis,revng,evaluation_of_irs,binrec,lifter_synthesis,superset_disassembly,b2r2,llvm_ir_lifter_using_mcsema_and_dyninst,gtirb,case_for_bap_and_angr,retrowrite}.

% The description must ensure that there is adequate preparation so that the degree project student does not need to do all their practical work without an established foundation, thus assuring time for conducting the scientific investigation.

\section{Eligibility and Study Planning}

\begin{quote}
	\textit{\textbf{Eligibility}: It is your responsibility, as a student, to verify that you are eligible to start your degree project. You must in your assignment description assure that you have completed all courses for the bachelors' degree and at least 60hp completed second cycle courses and that the 60hp include a course in scientific theory and method (e.g. DA2205, DA2210, DH2610, DM2572) and all courses relevant for the thesis.}
\end{quote}

I hereby assert that I have completed my Bachelor's degree in Computer Science, as received by Uppsala University. Furthermore, I assert that I have completed at least 60hp second cycle courses, including a course in scientific theory and method (the course code of which is \textbf{DA2210}: \textit{Introduction to the Philosophy of Science and Research Methodology for Computer Scientists}) and all courses relevant for this thesis (including \textbf{DD2481}: \textit{Principles of Programming Languages} and \textbf{DD2452}: \textit{Formal Methods}).

\begin{quote}
	\textit{\textbf{Study planning}: Also, you must list all courses remaining before you may apply for the master's degree and briefly describe how and when you plan to complete those courses.}
\end{quote}

The single remaining course I intend to finish before applying for the Master's degree is a course in Graph Theory (course code \textbf{SF2740}), to which I'm already registered and will be taking during HT19. I see no problem in completing this course along-side my degree project, and feel confident in the outcome.

% This is aimed to ensure that the thesis really is one of the last elements of your education.

% === [ Back matter ] ==========================================================

% --- [ References ] -----------------------------------------------------------

\bibliography{references}

% --- [ Appendices ] -----------------------------------------------------------

%% === [ Appendices ] ===========================================================

\appendix
\setcounter{secnumdepth}{0}
\section{Appendices}
\setcounter{secnumdepth}{3}
\renewcommand{\thesubsection}{\Alph{subsection}}

% TODO: add appendices.

%% === [ Foo ] ==================================================================

\chapter{Foo}

\todo{foo}

%\clearpage
%\input{sections/appendices/b_foo}
%\clearpage


\end{document}
